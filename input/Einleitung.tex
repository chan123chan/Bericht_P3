\section{Einleitung}

Das Thema Energieverbrauch wird in unserer Gesellschaft immer wichtiger. Um den Energieverbrauch zu reduzieren werden stets neue Ziele von einzelnen Ländern festgelegt. In der Schweiz zum Beispiel, soll der heutige Energieverbrauch um den Faktor drei reduziert werden. Um dies zu erreichen soll die sogenannte Standby Energie reduziert und die Wirkungsgrade der einzelnen Verbrauchskomponenten verbessert werden. Für dies ist es Notwendig den Verbrauch der einzelnen Komponenten über einen längeren Zeitraum nachvollziehen zu können. Um dies zu bewerkstelligen wird im Projekt 3 des Studienganges Elektrotechnik ein Leistungsmessgerät erstellt, welches nicht nur die momentane Effektivleistung anzeigt, sondern die Messdaten bis zu einer Woche aufzeichnen kann. 

Ziel ist es ein Leistungsmessgerät zu entwickeln, welches über eine Strommessung und eine Spannungsmessung die Leistung errechnen kann. Diese Daten gilt es drahtlos an ein multimediafähiges Gerät zu senden, von wo aus die aufgezeichneten Messdaten ausgelesen und grafisch dargestellt werden. Das Leistungsmessgerät muss für die normale Netzspannung ausgelegt sein und den spezifischen Sicherheitsnormen entsprechen.

Um die Leistungsmessung umzusetzen werden zwei Signale, das Spannungs-und Stromsignal, gemessen. Die Eingangssignale werden zuerst in einem analogen Schaltungsteil aufbereitet, danach an einem Mikrocontroller digitalisiert und schliesslich softwaremässig verarbeitet. Die errechnete Leistung wird zusammen mit einem Zeitstempel lokal abgespeichert. Diese Daten können auf Wunsch durch eine Bluetooth Verbindung an eine Java Applikation gesendet und dort visualisiert werden. Um die Sicherheit zu gewährleisten, wird ein verschlossenes Kunststoffgehäuse erstellt, welches keine Berührung an elektrisch leitendes Material zulässt.

Das Messgerät P3T7 kann die Leistung von Haushaltsgeräten messen. Das Gerät signalisiert mit Hilfe von 3 LEDS, in welcher Betriebssituation es sich befindet. Durch die Umschaltmöglichkeit von zwei Messbereichen ist es möglich eine genauere Messung zu erreichen. Die gemessenen Daten können durch ein mitgeliefertes Programm abgerufen und visualisiert werden. 
Der nachfolgende Bericht ist in fünf Teile aufgeteilt. Durch die Konzept Erklärung sowie den Technischen Grundlagen werden die Überlegungen für das Leistungsmessgerät dargelegt. In zwei Kapiteln werden Analog sowie Software Teil behandelt. Die Validierung sowie die Schlussfolgerungen werden zum Schluss beschrieben. 


\pagebreak