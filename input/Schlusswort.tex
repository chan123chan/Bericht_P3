\pagebreak
\section{Schlusswort}%Beni 1 Seite
P3T7 wurde während des dritten Semesters im Studiengang Elektrotechnik entwickelt. Der Prototyp, welcher gebaut wurde, ermöglicht es den Energieverbrauch von einphasigen Verbrauchern über einen Zeitraum von fünf Tagen aufzuzeichnen.  
 
Beim Erstellen des Prototyps wurde sich an den Zielen, welche im Kapitel 2 aufgelistet sind, orientiert.  
 
Die Sollziele der Hardware konnten bis auf die Peripherie umgesetzt werden. Auf ein «Betriebs-LED» wurde verzichtet, da das Gerät nur geöffnet werden darf, wenn es sich nicht am Netz befindet. Das Netzgerät entspricht der Schutzklasse zwei, aufgrund dessen benötigt es keine Potenzialtrennung. P3T7 wurde, um zu gewährleisten, dass es am Niederspannungsnetz angeschlossen werden kann, auf \^U=400V und \^I=15A ausgelegt. Um die Schaltung vor zu grossen Strömen zu schützen, besitzt die Schaltung 3 Sicherungen (Kap. 4.1).   
 
Die definierte Ungenauigkeit konnte für rein ohmsche Verbraucher eingehalten und im Messbereich von 5W bis 100W auf $\pm$ 0.5W, im Messbereich 100W bis 800W auf $\pm$1.5W und im Messbereich 800W bis 2300W auf $\pm$10W verbessert werden. Als Referenzmessung galt das Leistungsmessgerät Voltech PM1000. Es können Leistungen bis zu (5$\pm$0.5)W detektiert werden. Die Referenzmessungen befinden sich im Anhang. Die auf dem EEPROM gespeicherten Daten werden per Bluetooth an ein PC-Programm gesendet, welches die verbrauchte Wirkleistung aufzeigt. Ausserdem ist es möglich, die Daten als Textfile abzuspeichern. Zudem besitzt das Programm einen Echtzeit-Modus. In diesem wird pro Sekunde die aufgenommene Wirkleistung des Verbrauchers angezeigt.   
 
Bei nicht ohmschen Verbrauchern hat P3T7 grössere Abweichungen. Im Bereich von 5 bis 100W kann ein Fehler von $\pm$10 auftreten. Bei 100-2300W beträgt die Unsicherheit $\pm$50W. Dies ist auf das Umschalten der einzelnen Messbereiche zurückzuführen. Bei Strömen, welche nur kurze Spitzen besitzen, wird der Overflow später erkennt. Dies bedeutet, dass das Umschalten der Messbereiche zu spät geschieht. Durch dies entstehen die höheren Abweichungen. Zudem lässt der Mikrocontroller nur eine Sample-Zeit von 3’125Hz zu. So könnten Harmonische Schwingungen von maximal 1’420Hz detektiert werden.  
 
Durch einen leistungsfähigeren Mikrocontroller könnte man P3T7 verbessern, so wäre es möglich, höhere Oberwellen zu detektieren. Durch eine höhere Rechenleistung könnte man die einzelnen Effektiv-Werte berechnen. Dies würde das Umschalten der einzelnen Messbereiche verbessern und ermöglichen, die Scheinleistung sowie den Phasenverschiebungswinkel $\phi$ zu bestimmen. 
